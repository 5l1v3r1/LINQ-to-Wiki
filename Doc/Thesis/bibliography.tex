%%% Seznam použité literatury je zpracován podle platných standardů. Povinnou citační
%%% normou pro bakalářskou práci je ISO 690. Jména časopisů lze uvádět zkrácená, ale jen
%%% v kodifikované podobě. Všechny použité zdroje a prameny musí být řádně citovány.

\def\bibname{Bibliography}
\begin{thebibliography}{12}
\addcontentsline{toc}{chapter}{\bibname}

%\bibitem{lamport94}
%  {\sc Lamport,} Leslie.
%  \emph{\LaTeX: A Document Preparation System}.
%  2. vydání.
%  Massachusetts: Addison Wesley, 1994.
%  ISBN 0-201-52983-1.

\bibitem{mediawiki}
 \emph{MediaWiki}.
 MediaWiki.org.
 \url{http://www.mediawiki.org/wiki/MediaWiki}.

\bibitem{mediawiki-api}
 \emph{MediaWiki API}.
 MediaWiki.org.
 \url{http://www.mediawiki.org/wiki/API}.
 
\bibitem{cs-in-depth}
  {\sc Skeet,} Jon.
  \emph{C\# in Depth}.
  2nd edition.
  Stamford: Manning, 2011.
  Part 3, C\#~3: Revolutionizing how we code.
  ISBN 978-1-935182-47-4.
  
\bibitem{roslyn}
 \emph{Microsoft “Roslyn” CTP}.
 MSDN.
 \url{http://msdn.microsoft.com/en-US/roslyn}.
 
\bibitem{linq-to-sql-functions}
 \emph{Data Types and Functions}.
 LINQ to SQL Reference, MSDN Library.
 \url{http://msdn.microsoft.com/en-us/library/bb386970}.

\bibitem{warren}
 {\sc Warren}, Matt.
 \emph{LINQ: Building an IQueryable Provider -- Part III}.
 The Wayward WebLog.
 \url{http://blogs.msdn.com/b/mattwar/archive/2007/08/01/linq-building-an-iqueryable-provider-part-iii.aspx},
 2 August 2007.
 
\bibitem{reflection-emit}
 \emph{Emitting Dynamic Methods and Assemblies}.
 MSDN Library.
 \url{http://msdn.microsoft.com/en-us/library/8ffc3x75}.

\bibitem{codedom}
 \emph{Dynamic Source Code Generation and Compilation}.
 MSDN Library.
 \url{http://msdn.microsoft.com/en-us/library/650ax5cx}.
 
\bibitem{guidelines-for-names}
 \emph{Guidelines for Names}.
 Design Guidelines for Developing Class Libraries, MSDN Library.
 \url{http://msdn.microsoft.com/en-us/library/ms229002}.
 
\bibitem{wikitools}
 \emph{python-wikitools}.
 Google Code.
 \url{http://code.google.com/p/python-wikitools/}.
 
\bibitem{wikifunctions}
 \emph{WikiFunctions}.
 Wikipedia.
 \url{http://en.wikipedia.org/wiki/Wikipedia:WikiFunctions}.
 
\bibitem{linq-to-wikipedia}
 \emph{Linq to Wikipedia}.
 CodePlex.
 \url{http://linqtowikipedia.codeplex.com/}.
 
\end{thebibliography}