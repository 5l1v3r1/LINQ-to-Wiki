\chapter{The LinqToWiki library}
\label{ltw}

The LinqToWiki library consists of one Visual Studio solution, that contains the following projects:

\begin{itemize}
\item LinqToWiki.Core
\item LinqToWiki.Codegen
\item LinqToWiki.Codegen.App
\item LinqToWiki.ManuallyGenerated
\item LinqToWiki.Samples
\end{itemize}

The LinqToWiki.Core project contains the core of the library:
types that access the \ac{API}, convert to and from the representation of data in the \ac{API},
represent parameters of various types of queries, represent query results
or those that process \ac{LINQ} expression trees.
This project can be used together with code generated using LinqToWiki.Codegen,
or with manually written code.

The LinqToWiki.Codegen project handles generating code based on information from the \texttt{paraminfo} module.
It contains types that represent the results of that module, process them, generate C\# code and compile this code.
This project also contains helper types for easier creating of Roslyn syntax trees.

The LinqToWiki.Codegen.App project compiles down to a simple console application called linqtowiki-codegen,
that uses functionality from the LinqToWiki.\allowbreak{}Codegen project.

The LinqToWiki.ManuallyGenerated project is a sample of how one could write code to access a wiki using LinqToWiki without using LinqToWiki.Codegen to generate the code.

Finally, the LinqToWiki.Samples project contains samples showing how to use various API modules using LinqToWiki.
It uses code generated by LinqToWiki.\allowbreak{}Codegen.\allowbreak{}App.

\paragraph{Usage}

The intended usage of LinqToWiki is this:
First run the linqtowiki-codegen application to generate a \ac{DLL} tailored for a certain wiki.
Then use the generated library together with LinqToWiki.Core in your C\# (or \ac{VB.NET}) application to access that wiki.

An alternative is to get the generated \ac{DLL} from someone else and then use that.
An advantage of this approach is that the user does not have to have Roslyn installed.

Other options are possible, though.
For example, the LinqToWiki.Codegen library can be used to generate the code as a set of files containing C\# source code.
Those files can then be modified and manually compiled.

\section{The LinqToWiki.Core project}

The LinqToWiki.Core project contains shared code that can be used when querying any MediaWiki wiki
that has the \ac{API} enabled.
It can be used together with code generated through LinqToWiki\allowbreak{}.Codegen,
but it can also be used without it.

In fact, LinqToWiki.\allowbreak{}Codegen internally uses LinqToWiki.Core to access the \texttt{paraminfo} module
using manually written code.

\subsection{\texorpdfstring{\lstinline{QueryTypeProperties}}{QueryTypeProperties}}

The \lstinline{QueryTypeProperties} class holds basic information about a “query type”,
which corresponds to an \ac{API} module.
This information includes the prefix this module uses in its parameters,
what type of module it is or mapping of its result properties to values accepted by the \texttt{prop} parameter.
It is also able to parse \ac{XML} elements this module returns.

\subsection{\texorpdfstring{\lstinline{WikiQuery}}{WikiQuery}}

Probably the most often used and certainly the most interesting queries are those using \texttt{list} query modules.
Such queries are represented in LinqToWiki by a group of types whose names start with \lstinline{WikiQuery}.

Specifically, there are four such types:
\lstinline{WikiQuery}, \lstinline{WikiQuerySortable}, \lstinline{Wiki}\lstBreak\lstinline{Query}\lstBreak\lstinline{Generator} and \lstinline{WikiQuerySortableGenerator}.
If a module supports sorting, it is represented by a type with \lstinline{Sortable} in its name
and if it supports being used as a generator for \texttt{prop} queries, it is represented by a type with \lstinline{Generator} in its name.

There is also a fifth type: \lstinline{WikiQueryResult}.
This type by itself represents a query that cannot be modified anymore,
but can be used to execute it and get the results.
All of the four preceding types inherit from \lstinline{WikiQueryResult},
so it is possible  to execute the query using any one of them too.

The type governs what operations are available.
For example, if a type is one of the two \lstinline{Sortable} types,
it will have an \lstinline{OrderBy()} method, but no other type has this method.
Each method can also return a different type, as is necessary to form queries.

\paragraph{\lstinline{WikiQuery} and generics}

All of the \lstinline{WikiQuery}-related types are generic
and their type parameters are used to decide what properties can be used in each operation.
For example, the type parameter \lstinline{TOrderBy} of \lstinline{WikiQuerySortable}
decides what properties can be used in the parameter of the \lstinline{OrderBy} method.

The way this is achieved is that \lstinline{TOrderBy} is a type that contains the properties that can be
used for sorting in the module \lstinline{WikiQuerySortable} represents
and the \lstinline{OrderBy} method accepts lambda expressions whose parameter is of this type.

For example, if some module supported sorting by \lstinline{PageId} and \lstinline{Title},
then \lstinline{TOrderBy} would be a type that contains two properties with those names.
Because of this, a query like \lstinline{source.OrderBy(x => x.Title)} would compile and execute fine,
but \lstinline{source.OrderBy(x => x.Name)} would fail to compile.

Because of the way lambda expressions work, queries like \lstinline[breaklines=true]{source.OrderBy(x => x.Title.Substring(1))} or \lstinline{source.OrderBy(x => random.Next())} would compile fine.
But because there is no way to efficiently execute such queries using the MediaWiki \ac{API},
they will fail with an exception at runtime.
This is a well-known problem with LINQ that also affects LINQ to SQL \cite{linq-to-sql-functions}
and other LINQ providers.

\paragraph{\lstinline{WikiQuery} operations}

The standard query operators available on the \lstinline{Wiki}\lstBreak\lstinline{Query} types are:

\begin{itemize}
\item \lstinline{Where()} only sets some parameter or parameters of a query,
it always returns the same type.

It is available on all four of the basic \lstinline{WikiQuery} types
and uses the generic type parameter \lstinline{TWhere}.

\item \lstinline{Select()} is used to choose how the elements in the resulting collection should look like
and what properties should they contain.
Because the result of the lambda passed into this method can be an arbitrary type,
it doesn't make sense to modify the query after calling this method.
Because of that, \lstinline{Select()} returns \lstinline{WikiQueryResult}.
This also follows query expression syntax, where \lstinline{select} is the last clause of each query.

It is available on all four of the \lstinline{WikiQuery} types
and uses the type parameter \lstinline{TSelect}.

\item \lstinline{ToEnumerable()} and \lstinline{ToList()} are used to actually execute the query.
The distinction between the two methods is that \lstinline{ToEnumerable()} returns an \lstinline{IEnumerable},
that lazily loads new pages of results on demand.
\lstinline{ToList()}, on the other hand, returns a \lstinline{List},
that is immediately loaded with all of the results, possibly from many pages.

These two methods are available on all of the \lstinline{WikiQuery} types, including \lstinline{WikiQueryResult}
and return the result based on the type parameter \lstinline{TSource} for most of the types.
An exception is \lstinline{WikiQueryResult}, which uses a separate \lstinline{TResult} type parameter.

\item \lstinline{OrderBy()} (and \lstinline{OrderByDescending()}) sets the ordering.
Because it does not make sense to sort the same query multiple times
and because no module supports sorting by multiple keys,
this method returns the type with \lstinline{Sortable} removed.

This method is available on the two \lstinline{Sortable} types
and uses the type parameter \lstinline{TOrderBy}.

\item \lstinline{Pages} is a property that returns a \lstinline{PagesSource}
that can then be used in a \texttt{prop} query.
See Section~\ref{PagesSource} for more information.

This property is available on the two \lstinline{Generator} types
and uses the type parameter \lstinline{TPage}.

\end{itemize}

For a state diagram of transitions between the \lstinline{WikiQuery} types and other related types,
see Figure~\ref{WikiQuery types}.

\begin{figure}[htbp]
\begin{center}
\begin{tikzpicture}[>=angle 90]
\path (0,9) node(WQS) {\lstinline{WQSortable}};
\path (5,9) node(WQ) {\lstinline{WQ}};

\path (5,6) node(WQR) {\lstinline{WQResult}};

\path (0,3) node(WQSG) {\lstinline{WQSortableGenerator}};
\path (5,3) node(WQG) {\lstinline{WQGenerator}};

\path (0,0) node(PS) {\lstinline{PagesSource}};
\path (5,0) node(WQPR) {\lstinline{WQPageResult}};

\draw[->] (WQS) edge [out=188,in=172,looseness=5,auto] node {\lstinline{Where}} (WQS);
\draw[->] (WQS) edge node[above] {\lstinline{OrderBy}} (WQ);
\draw[->] (WQS) edge node[below left] {\lstinline{Select}} (WQR);

\draw[->] (WQ) edge [out=-20,in=20,looseness=8,right] node {\lstinline{Where}} (WQ);
\draw[->] (WQ) edge node[right] {\lstinline{Select}} (WQR);

\draw[->] (WQSG) edge [out=185,in=175,looseness=5,auto] node {\lstinline{Where}} (WQSG);
\draw[->] (WQSG) edge node[auto] {\lstinline{OrderBy}} (WQG);
\draw[->] (WQSG) edge node[above left] {\lstinline{Select}} (WQR);

\draw[->] (WQG) edge [out=-8,in=8,looseness=5,right] node {\lstinline{Where}} (WQG);
\draw[->] (WQG) edge node[right] {\lstinline{Select}} (WQR);

\draw[->] (WQSG) edge node[auto] {\lstinline{Pages}} (PS);
\draw[->] (WQG) edge node[auto] {\lstinline{Pages}} (PS);

\draw[->] (PS) edge node[auto] {\lstinline{Select}} (WQPR);

\end{tikzpicture}
\end{center}

\caption[State diagram of \lstinline{WikiQuery}-related types]
{State diagram of \lstinline{WikiQuery}-related types \\ (\lstinline{WikiQuery} is shortened to \lstinline{WQ} to save space)}
\label{WikiQuery types}

\end{figure}

\subsection{\texorpdfstring{\lstinline{PagesSource}}{PagesSource}}
\label{PagesSource}

The \lstinline{PagesSource} type represents a collection of pages that can be used in \texttt{prop} queries,
to get information about those pages.
This information can be for example a list of categories for each page in the collection.

There are two kinds of \lstinline{PagesSource}s: generator-based and list-based:

\begin{itemize}

\item List-based sources use a static list of pages, given as a collection of page titles, page IDs or revision IDs.

Because the number of pages given this way in a single \ac{API} request is fairly limited (usually to 50),
large lists have to be queried multiple times.
\lstinline{PagesSource} handles this transparently, so the user can input as many pages as he wants and does not have to worry about the limit.

One exception is if the limit is different than the default of 50 for the current user on the current wiki.
In that case, the user should change the limit by setting the property \lstinline{PagesSourcePageSize} on the \lstinline{Wiki} object.%
\footnote{In all other cases where limits are important in this library, they limit the output, not the input.
That is why simply setting \texttt{limit=max} works in those other cases, but does not work here.}

If the collection used to create a \lstinline{PagesSource} is lazy, it is iterated in a lazy manner.
For example, it could be the result of another LinqToWiki query, with additional processing by \ac{LINQ} to Objects,
that is not possible using LinqToWiki alone.
Or it could the result of a query from another wiki.
In such cases, the original query will only make as many requests as necessary for the follow-up query.

\item Generator-based sources represent a dynamic list of pages that is the result of another \ac{API} query,
like the list of all pages on a wiki from the \texttt{allpages} module.
This way, the list of pages does not have to be retrieved separately, only to be sent back.

Generator queries also have to handle paging, as described in Section~\ref{mw paging},
including the exception for the \texttt{revisions} module.

\end{itemize}

Thanks to the fact that both kinds of page sources for \texttt{prop} queries are represented by the same
(abstract) type, the user of this library can use the same code to work with any source,
thus avoiding repetitive code.

\paragraph{Structure of \texttt{prop} query}

To actually create a \texttt{prop} query for a page source, one uses the \lstinline{Select()} method.
Its parameter is a lambda, whose parameter is the type parameter \lstinline{TPage} of \lstinline{PagesSource}.
This type is the same for all queries on the same wiki, but could be different for different wikis.

Inside the lambda, properties and methods of the \lstinline{TPage} type can be accessed.
Each of them represents a \texttt{prop} module and all of the methods return one of the \lstinline{WikiQuery} types,
which can then be queried as usual, with one condition:
the \lstinline{WikiQuery} types can't ``leak'' outside of the query, so one has to use \lstinline{ToEnumerable()} or \lstinline{ToList()} inside the lambda.

There is a special case for the \texttt{revisions} module:
it can be also used with the \lstinline{FirstOrDefault()} method,
which means only the most recent revision for each page is selected.

If a \texttt{prop} module has a single result (not a collection), it is represented as a property
that directly returns this result, no querying is possible.

The methods of these \texttt{prop} queries are inside a lambda expression,
so they are not actually executed unless the expression was compiled and the resulting delegate invoked.
Because of this, processing them is not as simple as with normal queries.
For more details, see Section~\ref{PageExpressionParser}.

\medskip

For an example of \lstinline{PagesSource} query, see Figure~\ref{PS query}.

\begin{figure}[htbp]

\begin{lstlisting}
pagesSource.Select(
    p =>
    new
    {
        p.Info,
        Categories =
            p.Categories()
            .Where(c => c.Show == Show.NotHidden)
            .Select(c => new { c.Title, c.SortKeyPrefix })
            .ToEnumerable()
            .Take(10)
	}
)
\end{lstlisting}

\caption{Example of \lstinline{PagesSource} query \\ that uses \texttt{info} and \texttt{categories} modules}
\label{PS query}

\end{figure}

\needspace{5\baselineskip}

\subsection{\texorpdfstring{\lstinline{QueryParameters}}{QueryParameters}}

The \lstinline{QueryParameters} type contains the parameters of a query:

\begin{itemize}
\item sort direction and parameter by which to sort,
\item list of properties to select and a delegate that uses them to construct the result object,
\item list of other parameters, as key-value pairs.
\end{itemize}

\lstinline{QueryParameters} is an immutable type,
so that an initial subquery can be safely used repeatedly, as is the case with LINQ to Objects.
The list of other parameters is a functional-style immutable linked list.

\medskip

The \lstinline{PropQueryParameters} type derives from \lstinline{QueryParameters}
and is used to store information about a single module in a \texttt{prop} query.
Apart from inherited members, it also contains the name of the module
and a special value indicating whether to retrieve only the first item,
which corresponds to the usage of the \lstinline{FirstOrDefault()} method.

A related type is \lstinline{PageQueryParameters},
which represents a whole \texttt{prop} query.
That means it contains a list of \lstinline{PropQueryParameters} objects
and also information about the source of the query.

\subsection{\texorpdfstring{\lstinline{ExpressionParser}}{ExpressionParser}}

The \lstinline{ExpressionParser} static class is used to process expression trees from \ac{LINQ} methods
and store the processed query parameters in \lstinline{QueryParameters}.

Common for all expression tree processing is that closed-over local variables contained in the processed lambda,
which are represented as members of a compiler-generated closure class,
have to be first replaced by their actual value.
This is done using \lstinline{PartialEvaluator} written by Matt Warren \cite{warren}.

Also, some property names have to be translated from their C\# version to their \ac{API} version.
For details and the reason why this is necessary, see Section~\ref{ltwc-naming}.

\medskip

Each of the methods requires different processing. Specifically:

\begin{itemize}
\item Expression trees from \lstinline{Where()} are first split into one or more subexpressions
that are \emph{and}ed together (\lstinline{x => subexpr1 && subexpr2 && ...}; \emph{or} is not supported by the \ac{API})
and each of the subexpressions is then added as a key-value pair to the result.

Each subexpression has to be in the form \lstinline{x.Property == Value},
where \lstinline{Value} is a constant, possibly from an evaluated closed-over variable.
The reverse order (\lstinline{Value == obj.Property}) is also allowed.
An alternative for boolean properties is accessing the property directly (\lstinline{x.Property})
or negated (\lstinline{!x.Property}).

\item Processing \lstinline{OrderBy()} expression trees is simple:
they can either be identities (\lstinline{x => x}), which means default sorting will be used
(which is the only possibility for some modules),
or they can be simple property accesses (\lstinline{x => x.Property}),
which means the result will be sorted by that property.

The order of sorting (ascending or descending) is decided by the method used:
whether it was \lstinline{OrderBy()} or \lstinline{OrderByDescending()}.

\item Expression trees from \lstinline{Select()} are processed in two steps.
First, the expression is scanned for usages of its parameter.
If any of its properties are used, it means those properties have to be retrieved from the \ac{API}.
If the parameter is used directly, without accessing its properties,
it means all of the properties have to be retrieved, because it is impossible to say which of them will be used.

For example, the expression \lstinline|x => new { x.Property1, x.Property2 }|
means only \lstinline{Property1} and \lstinline{Property2} have to be retrieved.
On the other hand, \lstinline{x => SomeMethod(x)} means all of the properties have to be retrieved.

Second step is compiling the expression into a delegate,
which will then be executed for each item coming from the \ac{API}.

Put together, these two steps mean that \lstinline{Select()} can be used with any expression
and only properties that are actually needed will be returned by the \ac{API}.
\end{itemize}

\subsection{\texorpdfstring{\lstinline{PageExpressionParser}}{PageExpressionParser}}
\label{PageExpressionParser}

The class \lstinline{PageExpressionParser} is used to process
the \lstinline{Select()} lambda in \lstinline{PagesSource} queries.
The difficulty there is that the direct approach of building the query step-by-step,
used in normal queries, will not work.
That is because the expression has to be analyzed before there is any page object
that it expects as its parameter.

The result of this analysis is twofold:
the set of parameters needed for all of the \lstinline{prop} queries,
as a collection of \lstinline{PropQueryParameters},
and a delegate that can be used to get the result object for each page in the \ac{API} response.

\medskip

Because the subquery for each \texttt{prop} module has to end with a call to \lstinline{To}\lstBreak\lstinline{Enumerable()} or \lstinline{ToList()},
the parameters can be extracted by invoking the part of the subquery before that call.
At the beginning of each subquery is invoking a module-specific method on the page object.
But because there is no page object to use, that invocation is first replaced by an appropriate \lstinline{WikiQuery} object.

For example, for the query in Figure~\ref{PS query}, the invoked code is (where \lstinline{wikiQuery} is the appropriate \lstinline{WikiQuery} object):

\begin{lstlisting}
wikiQuery.Where(c => c.Show == Show.NotHidden)
	  .Select(c => new { c.Title, c.SortKeyPrefix })
\end{lstlisting}

\medskip

To get the delegate, all calls to \lstinline{Where()} and \lstinline{OrderBy()} are removed,
because their only purpose is to modify the query parameters.
Then the single parameter of type \lstinline{TPage} is replaced by a parameter of type \lstinline{PageData}
and calls to module methods are replaced by calls to \lstinline{GetData()},
with a type parameter specifying the type of the result and a parameter specifying the name of the module.

The \lstinline{GetData()} method returns a collection,
so for modules that return only a single item, like \texttt{info},
a call to \lstinline{Single}\lstBreak\lstinline{Or}\lstBreak\lstinline{Default()} is also added.

\needspace{11\baselineskip}

For example the expression in the query in Figure~\ref{PS query} is transformed into:

\begin{lstlisting}
pageData =>
new
{
    Info = pageData.GetData<InfoResult>("info")
    		     .SingleOrDefault(),
    Categories =
    	pageData.GetData<CategoriesSelect>("categories")
    		 .Select(c => new { c.Title, c.SortKeyPrefix })
    		 .Take(10)
}
\end{lstlisting}

\subsection{Other types}

The \lstinline{QueryProcessor} type manages downloading the result and transforming it from \ac{XML} to objects.
For queries whose result is a collection, it also handles returning the pages in a lazy manner
and downloading the follow-up pages when necessary.

The \lstinline{QueryPageProcessor} type does the same for \lstinline{PagesSource} queries.

\medskip

The \lstinline{Downloader} type takes care of forming the query string, executing the request and
returning the result as an \lstinline{XDocument}.
\lstinline{XDocument} is a part of LINQ to XML, a part of .NET Framework for manipulating \ac{XML} documents.

\lstinline{Downloader} always uses POST and formats its requests as \path{application/x-www-form-urlencoded}.
This means that all modules work, including those that require POST.
On the other hand, uploads of files do not work, because they require \path{multipart/form-data}.

The decision to use \path{application/x-www-form-urlencoded} follows from the fact that
\path{multipart/form-data} is very inefficient when sending multiple parameters with short values,
which is common when making requests to the \ac{API}.

% TODO: Namespace type?

\section{The LinqToWiki.Codegen project}
\label{ltwc}

The LinqToWiki.Codegen project contains code that retrieves information about API modules in some wiki,
then uses that information to generate C\# code to access those modules using Roslyn
and finally compiles the code into a library.

\medskip

Roslyn was chosen, because it is superior when compared with common approaches for code generation in .Net,
namely Reflection.Emit and CodeDOM.

Reflection.Emit \cite{reflection-emit} is a set of types that allow code generation of code at runtime.
The generated code can then be directly executed or saved as an assembly (.dll or .exe) to disk.
The distinguishing feature is that it uses the low-level Common Intermediate Language (CIL),
which means writing any code beyond the simplest methods can be very tedious and error-prone.

CodeDOM \cite{codedom} can be used to generate code and compile it to an assembly.
It uses language-independent model, which can be converted to various .Net languages,
including C\# and VB.NET.
This model is also the biggest disadvantage of CodeDOM, because it means it doesn't support all features of C\#.
For example, even such basic feature as writing a \lstinline{static} class is impossible in the CodeDOM model
without using “hacks”.

Detailed description of Roslyn is in Section~\ref{roslyn}.

\medskip

At this point, we have a library (LinqToWiki.Core) that can be used to access the MediaWiki API the way we want
from the final generated library.
We can also use the same library to get the information we need about the modules of the API from the \texttt{paraminfo} module.
And we have decided we want to use Roslyn to generate the final library.
What remains is to decide what code to generate, how exactly to map the modules, their parameters
and their results into the model of LinqToWiki.Core.

% think about moving this before LTW.Core

There are some decisions that were already made in LinqToWiki.Core
(the \texttt{sort} and \texttt{dir} parameters should map to \lstinline{OrderBy()};
the \texttt{prop} parameter maps to \lstinline{Select()}),
but several other decisions still remain:\footnote{
Obviously, both libraries were written alongside each other, to work well together, not one after the other.
But we think it's better to describe them this way, separately.}

\begin{itemize}
\item How should the remaining parameters be mapped?
Should they all go into \lstinline{Where()} or somewhere else? Where?
\item How should the modules that don't return lists be mapped?
LINQ methods are not suitable for them, because they are meant to work with collections.
\item How to name the generated types and members?
Specifically, how to represent names that can't be used (like those containing special characters)
and names that are undesirable (those that conflict with C\# keywords).
Also, should the generated members follow .Net naming conventions?
\end{itemize}

Our answers to these questions are in the following couple of sections.

\subsection{Naming of generated types and members}

Let us start with the last question: Should the generated members follow .Net naming conventions?
The .Net naming guidelines \cite{guidelines-for-names},
that are widely followed by various .Net libraries and the .Net framework itself,
state that names of types and public members should use PascalCase,
that is, each word of an identifier should start with a capital letter
and the identifier should not contain any delimiters (such as underscores).

We would prefer to follow these naming conventions, but, unfortunately, it is not possible.
That is because the names of modules, parameters, result properties and almost all enumerated types in the API
use names that are all lowercase, without delimiters between words.
That means there is no way to figure out which letters in an identifier should be capitalized
(apart from the first one).

As one of the more extreme examples,
one of the possible values of the \texttt{rights} parameter
of the \texttt{allusers} module on the English Wikipedia is \texttt{collection\-save\-as\-community\-page}.
A human can see that the proper name for that value using PascalCase would be \lstinline{CollectionSaveAsCommunityPage},
but a computer cannot.
(Actually, it is possible that the words could be reliably separated using natural language processing,
but doing that is outside the scope of this work.)

\medskip

Because different .Net languages have different sets of reserved identifier names
(usually, those are the language keywords)
and because libraries written in one language should be usable from other languages,
.Net languages provide a way to use their keywords as identifiers.
In the case of C\#, this is done by prefixing the identifier with an at sign.
So, for example, to use \texttt{new} as an identifier, one has to write \lstinline{@new}.

Thanks to this, using keyword-named identifiers is still possible,
although slightly less convenient than with normal identifiers.
Also, the naming guidelines suggest avoiding keywords as identifiers.

In MediaWiki core API modules, there are four identifiers that are also C\# keywords:
\texttt{namespace}, \texttt{new}, \texttt{true}, \texttt{false}.
Out of these, we decided to shorten \texttt{namespace} to \lstinline{ns},
which is a common abbreviation, so the meaning should not be lost.
The other three have to be written with \lstinline{@} (\lstinline{@new}, \lstinline{@true} and \lstinline{@false}) in C\#,
because we did not find a reasonable alternative for them.

\medskip

As for special characters, the delimiters hyphen (\texttt{-}), slash (\texttt{/}) and space
appear in some names in the API, but are not allowed in .Net identifiers,
so they are replaced by underscores (\texttt{\_}).

Some names also start with the exclamation mark~(\texttt{!}), to indicate negation.
Such names are translated by prefixing \lstinline{not_}.
So for example, \texttt{!minor} (which means that an edit is not a minor edit)
is translated into \lstinline{not_minor}.

One more special case is that some enumerated types allow an empty value.
Such value is then represented by the identifier \lstinline{none}.

\medskip

Another question is how to name the generated types.
There two kinds of generated types:
those that represent some enumerated type and those that represent parameters or results of some module.

For the latter kind, it is simple to come up with a convention like naming them by the module name,
suffixed by the specific kind of the type
(e.g. \lstinline{blockResult} for the result of the \texttt{block} module
or \lstinline{categorymembersWhere} for the type representing \lstinline{Where()} parameters for the \texttt{categorymembers} module).

But for the former kind, the situation is more complicated.
Enumerated types do not have names by themselves, they are part of a parameter or property that has a name.
The problem is that different modules often have parameters and properties with the same name,
while their type sometimes is the same and sometimes it is not.

So, there are two options: either let the types that look the same actually be the same generated type,
or let each parameter and property have its own distinct type.
If we merge the types that look the same, we should not use the module name in their name,
because one type can be used with different modules.
But that means we need to distinguish different types in another way, like a number.
But names like \lstinline{token5} are not very helpful for the user.

Because of that, we chose the other option, which means including the name of the module in the type name.
But doing it this way does not eliminate conflicts completely:
In the case when a module has a parameter and a property with the name,
their types still have to be distinguished.
An example of such type name is \lstinline{recentchangestype2}.

\subsection{Structure of generated code}
\label{ltwcg-structure}

At the start of each query is the \lstinline{Wiki} type.
It contains methods for non-query modules as well as methods to create list-based \lstinline{PagesSource}s.
It also contains the property \lstinline{Query} that returns an object that contains methods for
\texttt{list} and \texttt{meta} query modules.
(\texttt{prop} query modules work differently, for more information, see Section~\ref{PagesSource}.)

\medskip

With modules that don't return lists, the situation is mostly simple:
there are no parameters to sort or filter the result (because it's not a list)
and most of those modules also don't have parameters to choose the result properties.

Because of that, a method for each such module, that directly returns the result object is enough.
This method has parameters corresponding to the parameters of the module,
where required parameters of the module are mapped as normal method parameters
and parameters that are not required are mapped as optional parameters.
The code of this method builds \lstinline{QueryParameters} from the method parameters
and then executes the query using \lstinline{QueryProcessor}.

\medskip

On the other hand, list modules can have several kinds of parameters:

\begin{itemize}
\item Those that affect order of the items in the list. They are naturally mapped as \lstinline{OrderBy()}.
The parameters \texttt{sort} and \texttt{dir} belong here.
\item Those that choose what properties appear in the result. They are naturally mapped as \lstinline{Select()}.
Only the parameter \texttt{prop} belongs here.
\item Those that filter what items appear in the result. They are naturally mapped as \lstinline{Where()}.
For example, the parameters \texttt{namespace} and \texttt{startsortkey}
of the \texttt{categorymembers} module belong here.
\item Various other parameters. They do not naturally map to any LINQ method.
For example, the parameter \texttt{title} (that decides which category to enumerate)
of the \texttt{categorymembers} module belongs here.
\end{itemize}

The first two kinds are not a problem, because it is clear which parameters belong to them.
The second two kinds are a problem, because there is no clear way to automatically distinguish between the two.
One exception is if a parameter is required (as indicated in its description),
then it means it belongs to the other parameters.

Required parameters are given as parameters of the module methods,
but we decided to treat all non-required parameters that do not belong to the first two kinds,
as if they were \lstinline{Where()} parameters.
Unfortunately, this means that some queries do not logically make sense,
if we consider that the \lstinline{Where()} method should only filter the results.

For example, consider this query:

\begin{lstlisting}
wiki.Query.categorymembers()
    .Where(cm => cm.title == "Category:Query languages")
\end{lstlisting}

There, the \lstinline{title} property does not actually represent filtering
by the title of the category member, it decides which category to enumerate.
And without it, the query would not even execute successfully
(the parameter \texttt{title} is not marked as required, because the parameter \texttt{pageid} can be used instead of it).

Proper solution to this problem would require human interaction when generating the code,
to choose which parameters belong to \lstinline{Where()} and which do not.
As an alternative, the description of each parameter in the \texttt{paraminfo} module could contain its kind.

\medskip

One more question is how to represent enumerated types.
The answer is seemingly simple: make them \lstinline{enum}s
and for those parameters or properties that can have multiple values, use bit flags.
But the largest type that can be used as an underlying type for \lstinline{enum}
is \lstinline{ulong}, which has 64 bits.
That means this will work only if there is no enumerated type in the API,
that has more than 64 values and can have multiple values at the same time.
Unfortunately, the English Wikipedia has one:
the type of the \lstinline{rights} parameter of the \lstinline{allusers} module
has 106 values and the parameter can have multiple values at the same time.

Because of that, each enumerated type is represented by immutable class deriving from the common base class \lstinline{StringValue},
with inaccessible constructor and static field for each possible value.
Combination of values can be represented as a collection, like with other types.

\subsection{\texorpdfstring{\lstinline{Wiki}}{Wiki}}

The top-level type that manages all code generation is \lstinline{Wiki}
(not to be confused with the generated \lstinline{Wiki} type from Section~\ref{ltwcg-structure}).
It manages retrieving information about API modules and generating code for them.

When the code generation is complete, it saves the generated C\# files to a temporary directory
and compiles them using CodeDOM.
CodeDOM is used for the compilation,
because its compiler is the full C\# compiler and can handle all features of C\# (unlike the CodeDOM object model).
The Roslyn compiler is not able to compile some useful expressions, such as collection initializers
(but the object model of Roslyn is complete).

\subsection{\texorpdfstring{\lstinline{ModuleSource}}{ModuleSource}}

The \lstinline{ModuleSource} class is used to retrieve information about modules of the API
and transform it from XML to objects, like \lstinline{Module}, \lstinline{Parameter} and \lstinline{Parameter}\lstBreak\lstinline{Type}.
This information comes from the \texttt{paraminfo} module and is fetched using LinqToWiki.Core.

In fact, this code can be viewed as a sample on how to use LinqToWiki.Core
without code generated by LinqToWiki.Codegen.
Generated code cannot be used to work with the \texttt{paraminfo} module,
because it is one of the modules, whose response is complicated
and does not fit into the simple type system used by \texttt{paraminfo}.

Because the addition of result properties to \texttt{paraminfo} was made as a part of this work
(see Section~\ref{mw improvements}) and so is quite recent, there is also another option to get this information:
\lstinline{ModuleSource} can accept a “props defaults” file, that contains the necessary information.
The file looks the same as \texttt{paraminfo} response (in XML format),
except it contains only the added information.
This file can be created from another wiki that can already provide this information,
or it can be written by hand.
It can be also useful to work with modules from extensions, that currently don't provide this information.

\subsection{\texorpdfstring{\lstinline{ModuleGenerator}}{ModuleGenerator}}

\lstinline{ModuleGenerator} and related types are the ones that actually generate code for each module using Roslyn.
Each type generates code for a certain kind of module,
so for example \lstinline{ModuleGenerator} works with non-query modules,
while \lstinline{QueryModuleGenerator} works with most query modules.

Each generator creates all the code that is necessary for that module.
For example, for a \texttt{list} query module,
this includes generating \lstinline{Where}, \lstinline{Select} and possibly \lstinline{OrderBy} classes,
method in the \lstinline{QueryAction} class
(which is returned by the \lstinline{Query} property of the \lstinline{Wiki} class)
and types for all its enumerated types.

Each of the generated types and methods also has XML documentation comment attached,
based on description from \texttt{paraminfo}.
This means that a user of this library does not have to guess what each method or property means,
his IDE will show him description for it.

These descriptions sometimes contain references to details of the API that this library abstracts away.
For example, the description for the \texttt{unique} parameter of the \lstinline{alllinks} module says:
“Only show unique links. Cannot be used with generator or alprop=ids.”
The reference to \texttt{alprop} makes sense to someone who uses the API directly,
but would be very confusing for a user of LinqToWiki.
Not only does LinqToWiki abstract away module prefixes (\texttt{al}),
it also doesn't expose the \texttt{prop} parameter directly
(the \lstinline{Select()} method is used instead).

\subsection{\texorpdfstring{\lstinline{SyntaxEx}}{SyntaxEx}}

As described in Section~\ref{roslyn}, creating Roslyn syntax trees can be cumbersome.
The \lstinline{SyntaxEx} class makes doing that easier by adding simpler alternatives
to the factory methods in Roslyn's \lstinline{Syntax} class.
The \lstinline{SyntaxEx} methods do not handle more complex cases,
so for those, using \lstinline{Syntax} is still necessary.

For example of how the code from Figure~\ref{Roslyn code 2012} can be written using \lstinline{SyntaxEx},
see Figure~\ref{SyntaxEx code}.

\begin{figure}[htbp]

\begin{lstlisting}
SyntaxEx.AutoPropertyDeclaration(
  new[]
  {
    SyntaxKind.PublicKeyword,
    SyntaxKind.AbstractKeyword
  },
  "CategoryInfoResult",
  "CategoryInfo",
  setModifier: SyntaxKind.PrivateKeyword,
  isAbstract: true)
\end{lstlisting}

\caption{A sample code to manually create a Roslyn syntax tree \\ using \lstinline{SyntaxEx}}
\label{SyntaxEx code}
\end{figure}

\medskip

Another improvement is that syntax nodes that represent declaration of property, field, parameter or variable
can be used to refer to them in later code, for example when assigning the value of a parameter to a property.
This is achieved by using implicit conversions and a helper type \lstinline{NamedNode}.
In Roslyn without this extension, it is necessary to extract the name of the syntax node
and use that to create \lstinline{IdentifierNameSyntax}.

As with \lstinline{SyntaxEx}, this can make simple cases simpler, but cannot handle evything.
Because of that, complex cases still have to directly use Roslyn.

\section{The linqtowiki-codegen application}

\section{Samples of queries}

The project LinqToWiki.Samples contains one class with methods
that show the usage of each available module of the API.
It also contains one real-world complex query that combines LinqToWiki with LINQ to objects
to search for empty categories that are not redirects.

When run, it is a console application that shows the output of the selected module.
Selecting the module to use is done by changing which method is called by the \lstinline{Main()} method.

See Figure~\ref{query-samples} for few of the sample queries.

\begin{figure}[htbp]

\begin{lstlisting}
private static void Compare(Wiki wiki)
{
	var result = wiki.compare(fromrev: 486474789,
                                     torev: 487063697);
	Console.WriteLine(result.value);
}

private static PagesSource<Page> CategoryMembersSource(Wiki wiki)
{
	return (from cm in wiki.Query.categorymembers()
		 where cm.title == "Category:Query languages"
		     && cm.type == categorymemberstype.subcat
		 select cm).Pages;
}

private static void Images(PagesSource<Page> pages)
{
	var source = pages
		.Select(
			p =>
			PageResult.Create(
				p.info,
				p.images().ToEnumerable())
		);

	Write(source);
}

private static void QueryPage(Wiki wiki)
{
	var result = wiki.Query.querypage(
		querypagepage.Uncategorizedpages);

	Write(result);
}        
\end{lstlisting}

\caption{Some of the sample queries}
\label{query-samples}
\end{figure}