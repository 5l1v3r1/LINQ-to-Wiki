\section{LINQ and expression trees}

LINQ, short for Language Integrated Query, is a feature of the C\# programming language%
\footnote{Visual Basic .Net also supports LINQ, with slightly different syntax
and capabilities, but uses the same types.}
and the .Net framework that can be used for querying of various data sources
and appeared in the version 3.0 of the language \cite{cs-in-depth}.
It uses higher-order functions and lambda expressions to achieve a readable declarative syntax.

LINQ consists of a set of so called “standard query operators”:
methods that are used to perform the query operations on a given source.
Also, a special syntax (called “query expressions”), similar to SQL queries, is available
for some of those operators.
The compiler translates a query expression into a set of calls to standard query operators,
using lambda expressions and anonymous types.

Anonymous types are types that don't have to be explicitly declared;
they are used in similar situations as tuples in functional programming.
An instance of an anonymous type is created by using the \lstinline{new} keyword
without specifying the type of the object to create.

For example the following query expression (as seen in Chapter \ref{goal}):

\nopagebreak

\begin{code}
from product in allProducts
where product.Price > 500
   && product.InStock
join category in categories on product.Category equals category
orderby product.Price
select product.Name
\end{code}

Is translated into the following method calls:

\begin{code}
allProducts
    .Where(product => product.Price > 500 && product.InStock)
    .Join(
        categories,
        product => product.Category,
        category => category,
        (product, category) => new { product, category })
    .OrderBy(t => t.product.Price)
    .Select(t => t.product.Name)
\end{code}

The parameter~\lstinline,t, is called “transparent identifier”. It is used to transfer a set of variables from one method call to another.

The LINQ library also contains methods that do not have a corresponding representation in query expressions. Some examples of those are \lstinline,Aggregate(),, \lstinline,Sum(), and \lstinline,ToList(),.

Many of the basic query operators also correspond to a well-known higher-order functions from functional programming. See Figure \ref{LINQ methods} for comparison of some of the LINQ query operators, query expression clauses, and higher-order functions.

\begin{figure}[htbp]
\begin{tabular}{lll}
query operator & query expression clause & higher-order function \\
\lstinline,Select(), & \lstinline,select,, \lstinline,let, & map \\
\lstinline,Where(), & \lstinline,where, & filter \\
\lstinline,SelectMany(), & second and following \lstinline,from, & bind \\
\lstinline,Aggregate(), & & fold \\
\lstinline,Join(), & \lstinline,join, & \\
\lstinline,OrderBy(),,\cr \lstinline,OrderByDescending(), & \lstinline,orderby, & \\
\lstinline,GroupBy(), & \lstinline,group by, & \\
\lstinline,Sum(), \\
\lstinline,First(), \\
\lstinline,ToList(), \\
\end{tabular}

\caption{Comparison between LINQ query operators, query expression clauses, and higher-order functions}
\label{LINQ methods}
\end{figure}

Usually, lambda expressions are compiled into normal methods and passed to the query operator methods as delegates
(which are similar to function pointers in~C or first-class functions in functional languages).
But this would not be suitable for querying of sources that are not in-memory collections.
This is because the query has to be translated into another form,
like an SQL query or a set of parameters for the MediaWiki API.

Because of this, a lambda expression in C\# can be also compiled into another form:
an expression tree.
Expression tree is an object that represents the given lambda expression
in a form similar to an abstract syntax tree.
This object can be programmatically accessed and manipulated,
which allows translation of LINQ queries into other forms, such as SQL queries.
An expression tree can also be compiled into a delegate.

For an example of an expression tree, see Figure \ref{Expression tree}.

\begin{figure}[htbp]
\begin{center}
\Treek[2]{4}{
 & & & \K{Binary: AndAlso} \B{dll}_{\textnormal{Left}} \B{dr}^{\textnormal{Right}} \\
 & \K{Binary: GreaterThan} \B{dl}_{\textnormal{Left}} \B{dr}^{\textnormal{Right}} & & & \K{Member: InStock} \B{d}^{\textnormal{Expression}} \\
 \K{Member: Price} \B{d}_{\textnormal{Expression}} & & \K{Constant: 500} & & \K{Parameter: product} \\
\K{Parameter: product}
}
\end{center}

\caption{The body of the expression tree for the lambda expression \lstinline,product => product.Price > 500 && product.InStock,}
\label{Expression tree}
\end{figure}

The .Net framework contains two implementations of the query expression pattern:
the interfaces \lstinline,IEnumerable<T>, and \lstinline,IQueryable<T>,.
This means that any object that implements one of these two interfaces can be used in a LINQ query.

These two types implement the query expression pattern completely,
so they can be used with any LINQ operator.
Other custom types can implement only part of the query expression pattern,
which would mean only a subset of the LINQ operators are available for such types.

The \lstinline,IEnumerable<T>, interface usually represents an in-memory collection,
so its implementation of the LINQ operators use delegates.
The \lstinline,IQueryable<T>, interface is usually used to represent a remote collection
(such as a table in a relational database),
so its version of the LINQ operators use expression trees.

The \lstinline,IQueryable<T>, interface doesn't perform any translation of expression trees
into the target query language.
What it does is to combine the whole query into one expression tree,
which is then passed to an implementation of \lstinline,IQueryProvider,.

The query provider is then responsible for processing the expression tree
and translating it into its target query language.
If the query is not valid, the query provider will throw an exception at runtime.