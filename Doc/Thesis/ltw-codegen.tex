\section{The LinqToWiki.Codegen project}
\label{ltwc}

The LinqToWiki.Codegen project contains code that retrieves information about API modules in some wiki,
then uses that information to generate C\# code to access those modules using Roslyn
and finally compiles the code into a library.

\medskip

Roslyn was chosen, because it is superior when compared with common approaches for code generation in .Net,
namely Reflection.Emit and CodeDOM.

Reflection.Emit \cite{reflection-emit} is a set of types that allow code generation of code at runtime.
The generated code can then be directly executed or saved as an assembly (.dll or .exe) to disk.
The distinguishing feature is that it uses the low-level Common Intermediate Language (CIL),
which means writing any code beyond the simplest methods can be very tedious and error-prone.

CodeDOM \cite{codedom} can be used to generate code and compile it to an assembly.
It uses language-independent model, which can be converted to various .Net languages,
including C\# and VB.NET.
This model is also the biggest disadvantage of CodeDOM, because it means it doesn't support all features of C\#.
For example, even such basic thing as writing a \lstinline{static} class is impossible in the CodeDOM model
without using various ``hacks''.

\medskip