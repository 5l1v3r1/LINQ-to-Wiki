\chapter*{Introduction}
\addcontentsline{toc}{chapter}{Introduction}

Wiki websites running on the MediaWiki software (such as Wikipedia) offer an API (Application programming interface)
for programmatic access to their database.
Since MediaWiki contains many functions, the API is quite extensive too: the core installation contains over seventy “modules”
and more are available through extensions. Each module accepts parameters in the form of key-value pairs
and returns a structured response in one of the possible formats, including XML.

Because of the size of the API, accessing it from some programming language is not very simple.
Two basic approchaches are possible: static and dynamic.

The dynamic approach is to create a thin library around the API modules:
let the user specify the names of the parameters and their types and return the response
in a dynamic manner, possibly as an associative array, or something like XML DOM.

This way, the user is responsible for the correctness of his query and for correct processing
of the response.
Also, it is hard to discover what are the possible parameters, what values can they have
and what form of response to expect.
This can lead to excessive use of the documentation or a ``trial and error'' approach.

The static approach is to create an extensive library that contains methods tailored for every module,
each returning a different statically-typed result.

This way, many of the errors the user could make will result in a compile-time error
and the development environment can also advise the user what options are available.

But this approach is also inflexible: if the user wants to use something the library
was not made for, he can't.
Differences like this can be caused by different versions of the software,
different set of installed extensions, or just by different configuration.

Another question with the static approach is how to represent the parameters in code.
Most modules have many optional parameters, and so presenting them to the user
in an understandable manner might be a challenge.

One more problematic part might be how to represent choosing which properties to return in the result.
A list of strings representing the chosen properties might be suitable for the dynamic approach, but not so much for the static one.

\bigskip

This work introduces the LinqToWiki library that tries to solve all those problems
in the form of a .Net library.

The dynamic vs. static issues are solved by automatically generating statically typed
code based on the metadata the API provides about itself.
The code generation is performed using Roslyn,
which is a new implementation of a compiler for the C\# language written in C\#.

The problems specific to the static approach are solved by using LINQ (Language integrated query):
a set of features of the C\# language and the .Net framework,
that is useful for representing queries and their translation into another form.