\chapwithtoc{Introduction}

Wiki websites running on the MediaWiki software (such as Wikipedia) offer an \ac{API}
for programmatic access to their database.
Since MediaWiki contains many functions, the \acs{API} is extensive too:
the core installation contains over seventy “modules” and more are available through extensions.
Each module represents a function available from the \acs{API}.
Modules accept parameters in the form of key-value pairs
and return a structured response in one of the supported formats, including \acs{XML}.

Because of the size of the API, accessing it from programming languages is not easy.
Two basic approaches are possible: static and dynamic:

\begin{itemize}
\item The dynamic approach is to create a thin library around the API modules:
let the user specify the names of the parameters and their types and return the response
in a dynamic manner, possibly as an associative array, or something like \acs{XML} \ac{DOM}.

This way, the user is responsible for the correctness of his query and for correct processing
of the response.
Also, it is hard to discover what the possible parameters are, what values can they have
and what form of response to expect.
This can lead to excessive use of the documentation or a “trial and error” approach.

\item The static approach is to create an extensive (or “thick”) library that contains methods tailored for every module,
each returning a different statically-typed result.

This way, many of the errors the user could make will result in a compile-time error
and the development environment can also advise the user what options are available.

But this approach is also inflexible: if the user wants to use something the library
was not made for, he cannot.
Differences like this can be caused by different versions of the software,
different sets of installed extensions, or just by different configuration.

Another question with the static approach is how to represent the parameters in code.
Most modules have many optional parameters, and so presenting them to the user
in an understandable manner might be a challenge.

One more problematic part is how to represent choosing which properties to return in the result.
A list of strings representing the chosen properties might be suitable for the dynamic approach, but not so much for the static one.
\end{itemize}

\medskip

This work introduces the LinqToWiki library and related tools that try to solve all those problems
using the C\# language and the .NET platform.

The dynamic vs. static issues are solved by automatically generating statically typed
code based on the metadata the API provides about itself.
The code generation is performed using Roslyn,
which is a new implementation of a compiler for the C\# language written in C\#.

The problems specific to the static approach are solved by using \ac{LINQ}:
a set of features of the C\# language and the .NET Framework,
that is useful for representing queries and their translation into another form.

The library also abstracts away some other aspects of the \ac{API}, like paging of the results.

\secwithtoc{Key contributions}

The key characteristics that make the LinqToWiki library novel, when compared with similar querying libraries are:

\begin{itemize}
\item Using different types in different LINQ operators
(specifically in \lstinline{select}, \lstinline{where} and \lstinline{orderby})
\item Allowing different operators for different queries
(i.e. allowing sorting and generator queries only for some modules and disallowing sorting twice in one query)
\item Using code generation to achieve statically-typed queries
\end{itemize}

Another important aspect of this work was creating a patch for MediaWiki to describe results of \ac{API} modules,
which was accepted by MediaWiki developers and is now running on all Wikimedia wikis, including Wikipedia.

\secwithtoc{Structure of this work}
Chapter~\ref{goal} explains what is the goal of the LinqToWiki library.
Chapter~\ref{background} describes libraries and \acp{API}
that were important for creating this work.
Specifically, those are the MediaWiki API (Section~\ref{mediawiki}),
LINQ (Section~\ref{linq}) and Roslyn (Section~\ref{roslyn}).
Chapter~\ref{mw improvements} talks about changes that were made to MediaWiki,
to make LinqToWiki possible.
Chapter~\ref{ltw} describes the library itself, and the projects it consists of.
Chapter~\ref{future work} mentions some ways in which the library could be further improved.
Finally, Chapter~\ref{conclusion} sums up the work by explaining what was achieved.

\medskip

Appendix~\ref{cd} describes the contents of the enclosed CD.