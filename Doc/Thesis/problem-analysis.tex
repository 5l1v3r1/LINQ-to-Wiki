\chapter{Problem analysis}
\label{goal}

The goal of the LinqToWiki library is to be able to express requests using the MediaWiki \ac{API}
in a way that is readable, discoverable, checked by the compiler for correctness as much as possible and also flexible with regards to changes.

This is achieved by generating classes specific for each module and using them in LINQ queries.

\paragraph{Querying data in C\#}

\ac{LINQ} is a way of querying various data sources from the C\# language.
The two most commonly used variants are LINQ to Objects, and various versions of LINQ for \ac{SQL} databases.
LINQ to Objects is used for querying in-memory data, like arrays.
There are several widely-used libraries for accessing \ac{SQL} databases using LINQ, including LINQ to SQL, LINQ to Entities and NHibernate.

In all versions of \ac{LINQ}, the queries look the same. For example:

\begin{lstlisting}
from product in products
where product.Price > 500
   && product.InStock
join category in categories on product.Category equals category
orderby product.Price
select product.Name
\end{lstlisting}

A query like this is translated into a sequence of method calls that take their parameters in the form of lambda expressions.
For example, the \lstinline{where} part of the above query is translated into:

\begin{lstlisting}
products.Where(product => product.Price > 500 && product.InStock)
\end{lstlisting}

The commonalities between LINQ to Objects and SQL LINQ libraries are that the full range of operators is available
and that all properties of the queried type are available in all of them.

\paragraph{Querying MediaWiki}

The situation with the MediaWiki \ac{API} is different in several ways:

\begin{enumerate}
\item It does not support queries represented by many of the LINQ operators, including \lstinline{join} and \lstinline{group by}.
\item Some of the modules do not support sorting, some do. Of those that do support sorting, some allow specifying the sort key, others only the direction.
\item The sets of properties that are available for filtering, sorting and selecting are all different.
\item There are modules used for queries about a set of pages. Those pages can be from a hard-coded list or a result from some other module.
\item There are also parameters that do not fit into the LINQ model well. Some of them are required, some are not.
\end{enumerate}

The goal is to be able to represent all valid queries, while invalid queries should cause a compile-time error.

Specifically, unsupported operators (like \lstinline{join} and \lstinline{group by}) should cause an error for all modules,
while the \lstinline{orderby} clause should cause an error only for the modules that do not support sorting.

Also, all operators should support only those properties that are actually supported by the \ac{API}.
So, for example for the \texttt{blocks} module, the following query should compile and execute fine:

\begin{lstlisting}
from block in wiki.Blocks()
where block.Ip == "8.8.8.8"
orderby block descending
select block.ById
\end{lstlisting}

This is because
\begin{compactitem}
\item limiting the query by the blocked \ac{IP} address,
\item sorting without specifying the key and
\item selecting the ID of the user who performed the block
\end{compactitem}
are all allowed, while the following query should cause three errors:

\begin{lstlisting}
from block in wiki.Blocks()
where block.ById == 1234
orderby block.Expiry descending
select block.Ip
\end{lstlisting}

Here,
\begin{compactitem}
\item limiting by the ID of the user who performed the block,
\item sorting by the expiration date and
\item selecting the \ac{IP} address
\end{compactitem}
are all impossible.
(Actually selecting the \ac{IP} address of the blocked user is possible,
but the information is contained in properties with different names.)

\section{Alternatives}
\label{alternatives}

Using statically typed methods and custom LINQ provider is not the only way a library like this
could be built in C\#.
Some of the alternatives include:

\begin{itemize}
\item Using strings for everything

This is probably the simplest way to write a library for MediaWiki API.
It means parameters and their names (including those specifying which module to use) are specified
as a collection of string key-value pairs.

For example, the query from the previous section would look something like:

\begin{lstlisting}
wiki.Query(new Dictionary<string, string>
    {
        { "list", "blocks" },
        { "bkip", "8.8.8.8" },
        { "bkdir", "older" },
        { "bkprop", "byid" }
    })
\end{lstlisting}

The main disadvantage of this approach is there is no checking of the query:
any mistake in the query won't be found out until the query is actually executed.
It also means autocompletion cannot help the user in finding out which parameters are available.

In such implementation, the result of the query will be also some sort of dictionary
(indexed by names of result properties), which has the same kind of problems as using strings for parameter names.

Some modifications are possible, for example having a special parameter for the module prefix
(\texttt{bk} in the above example), which would make the code less repetitive.
But such changes will not fix the fundamental problems with this approach.

\item Using \lstinline{dynamic}

C\# 4.0 supports dynamic typing using the special type \lstinline{dynamic}.
When using that, C\# acts similarly as dynamic languages.
This means methods and their parameters do not have to be declared before they can be used.

For example, the same query used previously could look like this using \lstinline{dynamic}:

\begin{lstlisting}
dynamic wiki = ...;
wiki.ListBlocks("bk", ip:"8.8.8.8", dir:"older", prop:"byid")
\end{lstlisting}

For this query to work,
the library does not need any knowledge about the \texttt{blocks} module or its parameters.

This query is much more succinct than the string version
(although part of that is because of extracting the prefix,
which could be done with strings too, as mentioned).
But it suffers from the same issues: mistakes cannot be detected at compile time
and autocomplete will not be able to help.

In this case, the returned object will be also dynamic,
with the same advantages and disadvantages.

One interesting consequence of using \lstinline{dynamic} is that it means LINQ cannot be used.
Some of the reasons for this are that dynamically invoked methods cannot have lambdas as their arguments
(at least not directly) and that expression trees do not support \lstinline{dynamic}.

\item Manually written code for each module

Here, code for each module would be written manually, instead of using code generation.
This would be a significant amount of code which would have to be updated with every change of the API.
Another problem is that each wiki can have different set of extensions, which add their own API modules,
so distinct versions for some wikis would have to be maintained.

The biggest advantage of this approach is that a human can understand the documentation of a module,
so they can write code for a module better than a code generator such as LinqToWiki.Codegen could.
This is because not all details of how a module works can be encoded into its machine readable description.

\item Using \lstinline{IQueryable} instead of custom LINQ provider

The \lstinline{IQueryable} interface is often used for implementing custom \ac{LINQ} sources,
especially those for querying \ac{SQL} databases.

The biggest disadvantage of using this approach is that \lstinline{IQueryable}
supports all LINQ operators and all operators use the same set of properties.
This is not suitable for MediaWiki \ac{API},
because it does not support several LINQ operators,
because ordering is available only for some modules
and because filtering, sorting and projection all use different sets of properties.

\end{itemize}