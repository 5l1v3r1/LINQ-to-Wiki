\chapter{Future work}

% mention < and >?

While the LinqToWiki library is fully functioning and could be considered complete,
there is still room for improvement.
Some possible improvements were already mentioned
(proper capitalization in generated code;
distinguishing \lstinline{Where()} parameters from other parameters),
but there are also other improvements that could have larger impact on the usage of the library:

\begin{itemize}
\item \emph{F\# implementation}

The recent version 3.0 of the .NET-based functional language F\# has its own alternative to \ac{LINQ},
called query expressions.
Query expressions use their own set of types, so they are incompatible with C\# \ac{LINQ},
but they offer similar capabilities.
Unlike C\#, they also support creating custom operators,
which could provide better syntax for \lstinline{PagesSource} queries.

Also, F\# 3.0 supports type providers, which is a feature for automatic creating of types
just before they are used in code, that is, before compilation.
This would simplify the workflow of using LinqToWiki, because the code generation would be automatic.

\item \emph{Asynchrony}

The current version of LinqToWiki is completely synchronous,
which means there is always a thread blocked, while an application waits for a network response.
Better support for asynchrony is the main feature of the new C\#~5.0,
but LinqToWiki could be made asynchronous even without it.
This could be useful especially in \ac{GUI} applications, where the main thread should not be blocked for long periods of time.

A big part of LinqToWiki is working with modules that return lists,
but there is no single idiomatic way of representing asynchronous collections,
with or without C\# 5.0 improvements.
There are several possibilities, including a lazy collection of \lstinline{Task}s,
\acs{Rx} \lstinline{IObservable} or \acs{TPL} Dataflow \lstinline{ISourceBlock}.

Another way how asynchrony could be used in LinqToWiki is if the following page was being retrieved
even before the user finished processing the preceding page.
This could be especially useful for \lstinline{PagesSource} queries,
because the following primary and secondary page could be fetched in parallel.

\item \emph{Better compile-time checking}

One of the features of Roslyn that is not currently used by LinqToWiki is Visual Studio integration.
What that means is that one can use Roslyn to write various Visual Studio extensions,
including enhanced compile-time checking.

In the case of LinqToWiki, this could be used to verify that queries that compile are actually correct.
If not, a custom error would show in the list of errors.
Such checking would be also useful for other \acs{LINQ} providers, such as LINQ to SQL,
but each provider has its own rules about what is an error and what is a correct query.

% extensions?

\end{itemize}