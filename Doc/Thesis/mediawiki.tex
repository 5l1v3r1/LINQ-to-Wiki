\sectionnobreak{MediaWiki API}

MediaWiki is an open source wiki system.
It is written in the PHP programming language and uses a relational database to store its data, usually MySQL.
It is maintained by the Wikimedia Foundation, who also runs some of the biggest wikis, including Wikipedia and Wiktionary.
It is also used by many others, including Wikia, who runs many small wikis for various interests and
the unofficial wiki of the Faculty of Mathematics and Physics, $\upomega\upiota\upkappa\upiota$.matfyz.cz.

\medskip

There are several ways to programmatically access the database of some MediaWiki wiki.
First, it's possible to directly access the database using SQL.
This usually requires access to the server that runs the database, so it's not available in many cases.
For Wikimedia wikis, a read-only access to most data, but excluding article texts,
is available for registered users of Toolserver, run by Wikimedia Deutschland.

Mostly specific to Wikimedia wikis is also another option: database dumps. These are files that contain
dumps of some tables of the wikis. Their disadvantages are that the newest dump is usually several days or weeks old
and that the files can be huge, which is impractical for getting information about a small number of pages.

Last, but not least, is the MediaWiki API \cite{mediawiki-api}.
It can be used to remotely access any MediaWiki wiki (unless the API is disabled in the configuration)
using the HTTP protocol.

\medskip

Parameters for an API request are given in the query string of a GET request or in the body of a POST request
(modules that perform modifications require the use of POST).
The body of the POST request is usually formatted as \path{application/x-www-form-urlencoded},
but file uploads require the use of \path{multipart/form-data}.

Some parameters can accept multiple values at once.
In these cases, the values are separated by a pipe character (\texttt{|}).

There are some parameters that are common to many modules:

\begin{itemize}
\item The \texttt{prop} parameter is used to determine what properties will be present in the response.
The values of this parameter don't map directly to the properties of the response,
so for example specifying \texttt{prop=ids} might cause the property \texttt{pageid} to appear in the result.
\item The \texttt{dir} parameter is used to determine the order of the results:
whether it should be ascending or descending.
\item The \texttt{sort} parameter decides which property will be used to order the results of the query.
\end{itemize}

The response can be in one of the several available formats, the most widely used ones are XML and JSON.

The representation of most data types in the response is nothing unusual:
\texttt{string}s are formatted as strings, \texttt{integer}s as decimal numbers,
\texttt{timestamp}s are formatted according to ISO 8601.
Only \texttt{boolean}s have a possibly unexpected representation:
if the property is \texttt{false}, it is not present in the result at all,
and if its value is \texttt{true}, it is represented as an empty string.

If there is some problem executing a request, for example if a parameter has an invalid value,
a warning will be returned along with the result of the operation.
In the case of a fatal problem, such as when the user doesn't have the right to perform an action,
an error is returned, without any results.

\medskip

The API is divided into modules and there are two kinds of modules:
“normal” modules (called “non-query modules” in this work) and query modules.

Non-query modules are usually used to perform some action.
For example the \texttt{edit} module can be used to edit a page
and the \texttt{block} module can be used to block another user (it can be used only by users with sufficient privileges).

Query modules are used for retrieving information about the wiki. There are three types of query modules:

\begin{itemize}
\item \texttt{list} modules: Return contents of various lists.
For example the \texttt{all\-categories} module can be used to list all categories on a wiki,
while the \texttt{categorymembers} module can be used to list members of a certain category.
\item \texttt{prop} modules: Return information about a set of pages.
For example, the \texttt{categories} module can be used to retrieve the categories for each page in a given set.
\item \texttt{meta} modules: Return meta information that are not directly associated with pages.
For example the \texttt{userinfo} module can be used to retrieve information about the currently logged-in user.
\end{itemize}

For \texttt{prop} modules, the set of pages they operate on can be specified directly using page titles or page IDs.

Another option is to use some other module (usually a \texttt{list} module) as a so called “generator”.
This way, one can for example retrieve all categories of pages in a specific category,
by using the \texttt{categorymembers} module as a generator for the \texttt{categories} module.

Because more than one module can be used in one request,
the parameters for each module are distinguished by using prefixes.
For example, the prefix for the \texttt{categorymembers} module is \texttt{cm}.
So, setting its \texttt{limit} parameter to the value of~5 can be achieved by
adding \texttt{cmlimit=5} to the query string of a GET request or to the body of a POST request.

The API is also extensible: MediaWiki extensions can add their own modules and modify some behavior of existing modules.

An example of an API request URI and a response in the XML format is in Figure \ref{API example}.

\begin{figure}[htbp]
\texttt{http://en.wikipedia.org/w/api.php}~\texttt{?}\
\texttt{format}~\texttt{=}~\texttt{xml}~\texttt{\&}
\texttt{action}~\texttt{=}~\texttt{query}~\texttt{\&}
\texttt{list}~\texttt{=}~\texttt{categorymembers}~\texttt{\&}
\texttt{cmtitle}~\texttt{=}~\texttt{Category:Query\%20languages}~\texttt{\&}
\texttt{cmprop}~\texttt{=}~\texttt{title}~\texttt{\&}
\texttt{cmtype}~\texttt{=}~\texttt{page}~\texttt{\&}
\texttt{cmdir}~\texttt{=}~\texttt{descending}~\texttt{\&}
\texttt{cmlimit}~\texttt{=}~\texttt{5}

\begin{lstlisting}[language=xml]
<?xml version="1.0"?>
<api>
  <query>
    <categorymembers>
      <cm ns="0" title="YQL (programming language)" />
      <cm ns="0" title="Yahoo! query language" />
      <cm ns="0" title="XQuery" />
      <cm ns="0" title="XPath" />
      <cm ns="0" title="XBase++" />
    </categorymembers>
  </query>
  <query-continue>
    <categorymembers cmcontinue="page|5842415345|572327" />
  </query-continue>
</api>
\end{lstlisting}

\caption{Example of API request and response}
\label{API example}
\end{figure}

\subsection{Paging}
\label{mw paging}

Because the results of the API queries can contain thousands and sometimes even millions of entries,
the responses are limited.
For most modules, the default limit (when it is not specified as a parameter) is ten entries per page
and the default maximum is 500~entries for normal users.
For users with the \texttt{apihighlimits} right, the limits are raised, usually to 5000~entries per page.

In the \texttt{limit} parameter, one can specify either the exact value,
or the special value \texttt{max}, which means the maximum allowed for the current user.

To get the data from the following page, one has to use a value specified in the \texttt{query-continue}
element in the result (see Figure \ref{API example} again).
The value in this element is a transparent identifier of the next page.

The advantage of this system when compared with the conventional paging systems of numbering pages
or using numeric offsets is that it avoids missing entries and duplicates when the result
changes while retrieving the pages.

The API has no notion of transactions, so it is not possible to get fully consistent results
that would correspond to an exact moment in time.
But thanks to this paging system, one can be certain that an entry that should be in the result set
during retrieving of all of the pages will actually be present in the result set exactly once.

\medskip

The situation gets more complicated when using a \texttt{prop} module with another module as a generator.
That is because both modules have their own paging.

When such a request is made, the first response will contain a limited number of items from the generator
and a limited number of results from the \texttt{prop} module for those items.
To retrieve the next set of items from the generator, one has to use the \texttt{query-continue} for the generator
(called “primary paging” in this work).
To retrieve the next set of results for the items from the first result,
one has to use the \texttt{query-continue} for the \texttt{prop} module (called “secondary paging” here).

For an example, see Figure \ref{paging}.
It shows how the paging might work when using the \texttt{allpages} module as a generator,
together with the \texttt{prop} module \texttt{categories}.
The \texttt{query-continue} elements are not shown in the figure.

\medskip

\begin{figure}[tbp]
\begin{center}
\begin{tikzpicture}
\path (0,12) node(11) {
\begin{minipage}{150pt}
\begin{lstlisting}[language=xml,frame=single]
<query>
  <pages>
    <page title="A" />
    <page title="B">
      <categories>
        <cl title="X" />
        <cl title="Y" />
      </categories>
    </page>
  </pages>
</query>
\end{lstlisting}
\end{minipage}
} (7,12) node(12) {
\begin{minipage}{150pt}
\begin{lstlisting}[language=xml,frame=single]
<query>
  <pages>
    <page title="A">
      <categories>
        <cl title="X" />
        <cl title="Y" />
      </categories>
    </page>
    <page title="B" />
  </pages>
</query>
\end{lstlisting}
\end{minipage}
} (11,12) node(13) {\dots};

\path (0,4) node(21) {
\begin{minipage}{150pt}
\begin{lstlisting}[language=xml,frame=single]
<query>
  <pages>
    <page title="C">
      <categories>
        <cl title="X" />
        <cl title="Y" />
      </categories>
    </page>
    <page title="D" />
  </pages>
</query>
\end{lstlisting}
\end{minipage}
} (7,4) node(22) {
\begin{minipage}{150pt}
\begin{lstlisting}[language=xml,frame=single]
<query>
  <pages>
    <page title="C">
      <categories>
        <cl title="Z" />
      </categories>
    </page>
    <page title="D">
      <categories>
        <cl title="X" />
      </categories>
    </page>
  </pages>
</query>
\end{lstlisting}
\end{minipage}
};

\path (0,0) node(31) {\dots};

\draw[->] (11) -- (12);
\draw[->] (12) -- (13);
\draw[->] (11) -- (21);
\draw[->] (21) -- (22);
\draw[->] (21) -- (31);

%TODO: divné mezery
\end{tikzpicture}
\end{center}

\caption{Example of primary and secondary paging}
\label{paging}

\end{figure}

The situation is even more complicated with the \texttt{prop} module \texttt{revisions}.
It can be used to retrieve information about revisions of pages, including their text
and it is the only module that can be used to get the text of a set of pages.

For other modules, when no \texttt{limit} parameter is specified, a default value is used (usually 10)
and a \texttt{query-continue} element is present in the response, to access the remaining items.

But for the \texttt{revisions} module, not specifying the \texttt{limit} parameter means that only the most
recent revision will be shown and no \texttt{query-continue} will be present.
Also, when \texttt{limit} is specified, the module can operate only on one page at a time,
so for example one has to set the \texttt{limit} of a module used as a generator to~1.

\subsection{The \texttt{paraminfo} module}
\label{paraminfo}

A special importance for this work has the \texttt{meta} query module \texttt{paraminfo}.
This module can be used to retrieve information about modules,
which is necessary for generating code to access those modules in a static fashion.

Before this work, the \texttt{paraminfo} module provided some general information about the module
and, most importantly, information about parameters, their data types and a short description,
useful as a documentation for the generated code.

The data type of a parameter is either a simple type (e.g. \texttt{integer} or \texttt{string}),
or an enumeration of possible values.

A shortened example of a response from the \texttt{paraminfo} module
for the \texttt{categorymembers} module is in Figure~\ref{paraminfo sample}.

For code generation in LinqToWiki, another piece of information is necessary:
knowing the properties of the response and how do they map to the values of the \texttt{prop} parameter.
For information about how we added them, see Chaper~\ref{mw improvements}.

\begin{figure}[p]

\begin{lstlisting}[language=]
<module name="categorymembers" prefix="cm" querytype="list"
  generator="" listresult="" description="List all pages in a ...">
  <parameters>
    <param name="title" type="string" 
      description="Which category to enumerate (required). ..." />
    <param name="pageid" type="integer" 
      description="Page ID of the category to enumerate. ..." />
    <param name="prop" default="ids|title" multi=""
      description="What pieces of information to include ...">
      <type>
        <t>ids</t>
        <t>title</t>
        <t>sortkey</t>
        <t>sortkeyprefix</t>
        <t>type</t>
        <t>timestamp</t>
      </type>
    </param>
    <param name="namespace" multi="" type="namespace"
      description="Only include pages in these namespaces" />
    <param name="continue" type="string"
      description="For large categories, give the value ..." />
    <param name="limit" default="10" max="500" type="limit" 
      description="The maximum number of pages to return." />
    <param name="sort" default="sortkey"
      description="Property to sort by">
      <type>
        <t>sortkey</t>
        <t>timestamp</t>
      </type>
    </param>
    <param name="dir" default="ascending"
      description="In which direction to sort">
      <type>
        <t>ascending</t>
        <t>descending</t>
      </type>
    </param>
  </parameters>
</module>
\end{lstlisting}

\caption{Shortened response of the \texttt{paraminfo} module \\ for the \texttt{category\-members} module}
\label{paraminfo sample}
\end{figure}