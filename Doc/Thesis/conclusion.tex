\chapwithtoc{Conclusion}
\label{conclusion}

The goal of this work was to implement a C\# library to access the MediaWiki \ac{API}
in a way that is readable, discoverable, strongly-typed and flexible.

These goals were successfully accomplished using a custom \acs{LINQ} provider and code generation with Roslyn \ac{CTP}:

\begin{itemize}
\item Code written using LinqToWiki is readable:
non-query modules are accessed using simple methods,
with parameters usually given as named parameters of the method.

Query modules use \acs{LINQ} methods or \acs{LINQ} query expressions,
which should be readable to any C\# programmer.
Although the fact that even parameters that do not filter the results go into \lstinline{Where()} can be confusing.

The syntax for \texttt{prop} modules, with \acs{LINQ} queries inside \lstinline{Select()},
usually following another \acs{LINQ} query, can be quite complicated,
but we believe it describes the meaning of the code quite well,
so it should still be fairly readable.

\item The various actions possible through LinqToWiki are highly discoverable:
every action begins in a single point – the \lstinline{Wiki} class.
An \ac{IDE}, such as Microsoft Visual Studio,
can then show the user the available actions, along with their description,
through autocompletion (called IntelliSense in Visual Studio).

\item The whole library is strongly-typed:
accessing modules, setting their parameters and then accessing their results
never involves using string constants,
that would represent some module, parameter or result property.
This is because actual types with methods, method parameters and properties are used.

Thanks to this, the chance for user error when using this library is greatly lowered.

\item The library is flexible: If some module in the \ac{API} of a wiki changes
(which happens regularly), the library should be still usable.
This is achieved by regenerating code for the wiki.

The same principle also applies to different wikis.
If the modules in several wikis differ, code can be generated for each wiki separately.
\end{itemize}

\medskip

Probably the biggest difference between the original goal and the final library is naming.
Names of types, methods, method parameters and properties should follow the .NET naming guidelines,
but this was not done, because we could not separate names from the \ac{API} into words.

%TODO some asctual conclusion?