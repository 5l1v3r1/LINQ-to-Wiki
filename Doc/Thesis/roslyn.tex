\section{Roslyn}

Microsoft Roslyn is a new implementation of the C\# compiler written in C\#
(and a VB.NET compiler written in VB.NET).
Its main distinguishing characteristic is that it is “open”:
it can be used as a library to convert between text and a syntax tree,
to manipulate the syntax tree or to interrogate semantic information.

It also integrates itself into the Microsoft Visual Studio IDE
(Integrated Development Environment), where it can be used to perform
custom refactoring actions or to produce custom errors and warnings at compile-time.

Roslyn is currently under development and its latest public version is a CTP
(Community Technology Preview) from October 2011.
In the CTP, the semantic part of the library in completely implemented,
so for example the syntax tree can represent any construct of C\#
and any syntax tree can be translated to and from source code.

On the other hand, the semantic part of the library is not fully implemented in the CTP,
which means that for example some syntax trees won't successfully compile,
even if they represent valid C\# code.

Because of its close relation with Visual Studio,
Roslyn syntax tree is able to represent every feature of C\# with down to character precision,
including “trivia”: parts of code that are not significant for the compiler,
such as whitespace and comments.

Trivia can also be “structured”, that is it can form a small syntax tree of its own.
An example of this structured trivia are XML documentation comments,
that can be used to provide documentation for a piece of code,
which can then be automatically processed.

For an example of a Roslyn syntax tree, see Figure~\ref{Syntax tree}

\begin{figure}[htbp]

\begin{lstlisting}
public abstract CategoryInfoResult CategoryInfo { get; }
\end{lstlisting}

\begin{center}
\Treek[.5]{4}{
& & & & \K{Property}\Below{Declaration} \B[-5]{dllll} \B[-5]{dll} \B[-5]{d} \B[-5]{drr} \B[-5]{drrrr} \\
\K{Public}\Below{Keyword} & & \K{Abstract}\Below{Keyword} & & \K{Identifier}\Below{Name} \B[-5]{d} & & \K{Identifier}\Below{Token:}\Below{CategoryInfo} & & \K{Accessor}\Below{List} \B[-5]{dll} \B[-5]{d} \B[-5]{drr} \\
& & & & \K{Identifier}\Below{Token:}\Below{CategoryInfoResult} & & \K{OpenBrace}\Below{Token} & & \K{GetAccessor}\Below{Declaration} \B[-5]{dl} \B[-5]{dr} & & \K{CloseBrace}\Below{Token} \\
& & & & & & & \K{Get}\Below{Keyword} & & \K{Semicolon}\Below{Token}
}
\end{center}

\caption{An example of a piece of C\# code and its Roslyn syntax tree \\ (trivia not shown)}
\label{Syntax tree}
\end{figure}

\pagebreak[0]

Roslyn syntax trees are immutable and they can be created using factory methods from the \lstinline{Syntax} class.
And while not all elements of the syntax tree have to be specified (like the braces of the accessor list),
creating a syntax tree can be quite cumbersome.

For an example of code to manually create the syntax tree from Figure~\ref{Syntax tree}, see Figure~\ref{Roslyn code}.

\begin{figure}[htbp]

\begin{lstlisting}
Syntax.PropertyDeclaration(
  modifiers:
    Syntax.TokenList(
      Syntax.Token(SyntaxKind.PublicKeyword),
      Syntax.Token(SyntaxKind.AbstractKeyword)),
  type: Syntax.ParseTypeName("CategoryInfoResult"),
  identifier: Syntax.Identifier("CategoryInfo"),
  accessorList:
    Syntax.AccessorList(
      accessors:
        Syntax.List(
          Syntax.AccessorDeclaration(
            SyntaxKind.GetAccessorDeclaration,
            semicolonTokenOpt:
              Syntax.Token(SyntaxKind.SemicolonToken)))));
\end{lstlisting}

\caption{A sample code to manually create a Roslyn syntax tree}
\label{Roslyn code}
\end{figure}