\section{Roslyn}
\label{roslyn}

Microsoft Roslyn is a new implementation of the C\# compiler written in C\#
(and a VB.NET compiler written in VB.NET) \cite{roslyn}.
Its main distinguishing characteristic is that it is “open”:
it can be used for example to convert between text and a syntax tree,
to manipulate the syntax tree or to interrogate semantic information.

It also integrates itself into the Microsoft Visual Studio IDE
(Integrated Development Environment), where it can be used to perform
custom refactoring actions or to produce custom errors and warnings at compile-time.

Roslyn is currently under development and so far it had two public releases.
Both were in the form of CTP (Community Technology Preview),
the first one from October 2011, the second one from June 2012.

In the second CTP, the syntactic part of the library in completely implemented,
so for example the syntax tree can represent any construct of C\#
and any syntax tree can be translated to and from source code.
On the other hand, the semantic part of the library is not fully implemented,
which means that for example some syntax trees won't successfully compile,
even if they represent valid C\# code.

Because of its close relation with Visual Studio,
Roslyn syntax tree is able to represent every feature of C\# with down to character precision.
This includes “trivia”: parts of code that are not significant for the compiler,
such as whitespace and comments.

Trivia can also be “structured”, that is, it can form a small syntax tree of its own.
An example of structured trivia are XML documentation comments,
that can be used to provide documentation for a piece of code,
which can then be automatically processed.

For an example of a Roslyn syntax tree, see Figure~\ref{Syntax tree}

\begin{figure}[htbp]

\begin{lstlisting}
public abstract CategoryInfoResult CategoryInfo { get; }
\end{lstlisting}

\begin{center}
\Treek[.5]{4}{
& & & & \K{Property}\Below{Declaration} \B[-5]{dllll} \B[-5]{dll} \B[-5]{d} \B[-5]{drr} \B[-5]{drrrr} \\
\K{Public}\Below{Keyword} & & \K{Abstract}\Below{Keyword} & & \K{Identifier}\Below{Name} \B[-5]{d} & & \K{Identifier}\Below{Token:}\Below{CategoryInfo} & & \K{Accessor}\Below{List} \B[-5]{dll} \B[-5]{d} \B[-5]{drr} \\
& & & & \K{Identifier}\Below{Token:}\Below{CategoryInfoResult} & & \K{OpenBrace}\Below{Token} & & \K{GetAccessor}\Below{Declaration} \B[-5]{dl} \B[-5]{dr} & & \K{CloseBrace}\Below{Token} \\
& & & & & & & \K{Get}\Below{Keyword} & & \K{Semicolon}\Below{Token}
}
\end{center}

\caption{An example of a piece of C\# code and its Roslyn syntax tree \\ (trivia not shown)}
\label{Syntax tree}
\end{figure}

\pagebreak[0]

Roslyn syntax trees are immutable and can be created using factory methods from the \lstinline{Syntax} class.
And while not all elements of the syntax tree have to be specified (like braces of a property accessor list),
creating a syntax tree can be quite cumbersome.

The exact syntax for creating syntax trees changed between the two CTPs.
In the October 2011 CTP, methods with many optional parameters were used.
In the June 2012 CTP, the situation somewhat improved:
the factory method now has parameters only for required children of the created node
and optional child nodes can be added in a fluent manner using \lstinline{With*} methods.

For an example of code to manually create the syntax tree from Figure~\ref{Syntax tree},
see Figure~\ref{Roslyn code 2011} for the October 2011 CTP version and
Figure~\ref{Roslyn code 2012} for the June 2012 CTP version.

\begin{figure}[p]

\begin{lstlisting}
Syntax.PropertyDeclaration(
  modifiers:
    Syntax.TokenList(
      Syntax.Token(SyntaxKind.PublicKeyword),
      Syntax.Token(SyntaxKind.AbstractKeyword)),
  type: Syntax.ParseTypeName("CategoryInfoResult"),
  identifier: Syntax.Identifier("CategoryInfo"),
  accessorList:
    Syntax.AccessorList(
      accessors:
        Syntax.List(
          Syntax.AccessorDeclaration(
            SyntaxKind.GetAccessorDeclaration,
            semicolonTokenOpt:
              Syntax.Token(SyntaxKind.SemicolonToken)))));
\end{lstlisting}

\caption{A sample code to manually create a Roslyn syntax tree \\ using October 2011 CTP}
\label{Roslyn code 2011}
\end{figure}

\begin{figure}[p]

\begin{lstlisting}
Syntax.PropertyDeclaration(
  Syntax.ParseTypeName("CategoryInfoResult"),
  "CategoryInfo")
  .WithModifiers(
    Syntax.TokenList(
      Syntax.Token(SyntaxKind.PublicKeyword),
      Syntax.Token(SyntaxKind.AbstractKeyword)))
  .WithAccessorList(
    Syntax.AccessorList(
      Syntax.List(
        Syntax.AccessorDeclaration(
          SyntaxKind.GetAccessorDeclaration)
          .WithSemicolonToken(
            Syntax.Token(SyntaxKind.SemicolonToken)))));
\end{lstlisting}

\caption{A sample code to manually create a Roslyn syntax tree \\ using June 2012 CTP}
\label{Roslyn code 2012}
\end{figure}